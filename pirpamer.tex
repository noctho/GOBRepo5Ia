\section{Trello}
Trello ist eine web-basierte Projektmanagementsoftware. Diese Software wurde von US-amerikanischen Unternehmen Fog Creek Software entwickelt. Trello nutzt MongoDB, Node.js und Backbone.js. Dieses Projekt wurde mit JavaScript verwirklicht.

\begin{figure}[h]
  \centering
    \includegraphics[width=\textwidth]{trello.png}
  \caption{Logo von Trello}
  \label{fig:trello}
\end{figure}
\subsection{Geschäftsmodell}
Geschäftsmodell von Trello basiert auf dem Freemium-Prinzip: Die Grundfunktionen sind kostenlos verfügbar. Jedoch Extrafunktionen können nur mit einen Gold-Account genutzt werden.
\subsection{Geschichte}
Im Sommer 2010 beginnt Fog Creek Software mit einem potenziellen Produkt her zum zu entwickeln. Im Januar 2011 wird ein Prototyp vorgeschlagen, der zur Lösung einiger hochrangiger Planungsprobleme dienen soll. Sein Name lautet Trellis. Kurz danach beginnt die Entwicklung in Vollzeit.

Nach abgeschlossener Betaphase startet Trello im September 2011 auf der TechCrunch Disrupt mit Apps für Internet und iPhone. Ab hier wurde der Name Trell gewählt.

Der Dienst wurde am 13. September 2011 gegründet und ist nur in English verfügbar. Aktuell hat Trello 7 Millionen Benutzer (Stand: Januar 2015). Der Entwichlung dieses Dienstes fand nach dem Vorbild der Produktionsplanungsmethode Kanban statt. Dies kann über ihre Webanwendung trello.com oder über ihre App die für iOS, Android und Windows 8 verfügbar ist.

Im Sommer 2012 wurde Taco, der Husky von Fog Creek-Mitbegründer Joel Spolsky, Maskottchen von Trello. Trello erreicht die Marke von 500.000 Mitgliedern und führt die Trello-Android-App ein.
\begin{figure}[h]
  \centering
    \includegraphics[width=0.7\textwidth]{taco.png}
  \caption{Maskottchen Taco}
  \label{fig:taco}
\end{figure}
\subsection{Nutzung}
Ihr Dienst läuft über ihre sogenannten Boards. Hier können Listen bearbeitet und modifiziert werden. Es können Checklisten, Anhänge und mit einem festgelegten Termin versehen werden. Den Zugriff auf diese sogenannten Boards kann beliebig vielen Nutzern zur Verfügung gestellt werden. Dies kann durch ein einfaches Drag and Drop gemacht werden. Diese können sich bei der Planung und der Kommunikation des Projekts beteiligen. Hier können an die Bretter verschiedene Probleme angesprochen und zusammen gelöst werden. Der Dienst bietet den Nutzern die Möglichkeit Kommentare zu hinterlassen, Dateien hochzuladen, Checklisten erstellen, Termine und viele weitere Features.
Daten können nicht nur vom Computer selbst sondern acuh von Google Drive, Droppbox oder OneDrive hinzugefügt werden. Um das ganze noch aufzupeppen sind \glqq emojis\grqq \space auch natürlich verfügbar. Um auf den laufenden über sein Projekt zu sein Wird man in der App selbst benachrichtigt, über E-Mail, Desktop Benachrichtigungen über den Browser oder über \glqq push notifications\grqq.

Hier ein kleines Video dazu: https://goo.gl/ehSahR

