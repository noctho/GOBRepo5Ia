\section{Bitbucket}
    
	Bitbucket ist ein webbasierter Hosting-Dienst für Software-Entwicklungsprojekte, der die Versionsverwaltungssysteme Git und Mercurial unterstützt.
    
	\subsection{Geschichte}

		Der Dienst wurde ursprünglich als reines Mercurial-System von Jesper Nøhr entwickelt und 2010 von dem australischen Unternehmen Atlassian gekauft und am 3. Oktober 2011 um Unterstützung für Git erweitert.
    
	\subsection{Eigenschaften}
    
		In Bitbucket wir nicht das Projekt als Sammlung von Quelltexten verwaltet, sondern die Repositorys der Nutzer. Gleichzeitig wird das Erstellen (Branchen) und Wiedervereinigen (Mergen) von Abspaltungen (Forks) besonders propagiert. Bei Bitbucket ist es wie gesagt möglich, den eigenen Code zentral zu verwalten und zu versionieren. Dabei stehen zwei Versionierungs-Plattformen für die eigenen Repositories zur Verfügung: Git und Mercurial. Zusätzlich ist auch das Folgen von anderen Bitbucket-Nutzern und Teams möglich.

Bitbucket ist ein Web-basierter Online-Dienst, d.h. er läuft auf einem Browser und damit überall. Die Repositories befüllt man wie gehabt mit Git oder Mercurial von einem lokalen System aus. Bitbucket bietet zudem dank einer API auch verschiedene Apps an. Auf Android existiert beispielsweise die App Bitbeaker.

	\subsection{Kosten}

		Bitbucket hat ein Premium-Modell. Die Basis-Nutzung ist kostenlos. Diese Basis-Nutzung umfasst eine beliebe Anzahl an öffentlichen und privaten Repositories mit allen dazu nötigen Features, wie Issues, Wikis etc. Teams können aus bis zu 5 Mitgliedern bestehen, welche sich Repositories teilen. Diese Anzahl kann man bis auf 8 Mitglieder erhöhen. Wer mehr Mitglieder in seinem Team verwalten möchte, muss den Dienst bezahlen. Bei 10 Nutzern sind es 10\$/Monat, bei 25 Nutzern sind 25\$/Monat usw. Beim Unlimited-Plan bezahlt man 200\$/Monat.
	\subsection{Bitbucket vs Github}

		\begin{itemize}

			\item Bitbuckets offensichtlichster Vorteil ist die Möglichkeit, kostenlos eine unbegrenzte Zahl an privaten Repositories mit bis zu fünf Beteiligten zu hosten.

			\item Bitbucket unterstützt nicht nur Git sondern auch Mercurial.
			\item In Github ist es möglich öffentliche kostenlose Repositorys zu erstellen.

		\end{itemize}
	\subsection{Verwendungen}
		2014 arbeiteten über 330.000 Teams aus über 2,5 Millionen Entwicklern mit Bitbucket, 200 Terabyte Code wurden gehostet. Zu den Unternehmen, die Gebrauch von Bitbucket machen, zählen neben Atlassian:
		\begin{itemize}
			\item DHL
			\item PayPal
 
			\item Teslaa Motos
			\item The New Yorks Times
		\end{itemize}
Auch die Open-Source-Projekte Eigen und OGRE sind bei Bitbucket gehostet.
	\subsection{Fazit}
    
		Bitbucket ist ein webbasierter Hosting-Dienst wie Github. Im grunde genommen ergänzen sich die beiden Programme jedoch recht gut, den bei Bitbucket kann man unbegrenzt viele private Repositorys erstellen und bei Github hingegen kann man unbegrenzt viele öffentliche Repositorys erstellen.
