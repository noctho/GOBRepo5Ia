\section{Bitucket}
Bitbucket ist ein webbasierter Hosting-Dienst für Software-Entwicklungsprojekte, der die Versionsverwaltungssysteme Git und Mercurial unterstützt. Der Dienst wurde ursprünglich als reines Mercurial-System von Jesper Nøhr entwickelt und 2010 von dem australischen Unternehmen Atlassian gekauft und am 3. Oktober 2011 umUnterstützung für Git erweitert.
\subsection{Eigenschaften}
Im Gegensatz zu anderen Open-Source-Hostern wie SourceForge ist auf Bitbucket nicht das Projekt als Sammlung von Quelltext zentral, sondern der Nutzer mit seinen Repositorys (Verzeichnissen, die vom jeweiligen VCS kontrolliert werden). Gleichzeitig wird das Erstellen (Branchen) und Wiedervereinigen (Mergen) von Abspaltungen (Forks) besonders propagiert. Forks dienen weiterhin dazu, einfach bei anderen Projekten mitentwickeln zu können.
Benutzer werden dazu ermutigt, in Teams zusammenzuarbeiten. Anders als bei anderen VCS-Diensten wie GitHub stellt Bitbucket grundsätzlich jedem Benutzer eine unbegrenzte Anzahl privater, also nicht öffentlich sichtbarer Repositorys kostenlos zur Verfügung, außerdem können Teams aus bis zu fünf Mitgliedern ein kostenfreies Abonnement abschließen. Größere Teams müssen einen monatlichen Betrag entrichten.
\subsection{Verwendung}
2014 arbeiteten über 330.000 Teams aus über 2,5 Millionen Entwicklern mit Bitbucket, 200 Terabyte Code wurden gehostet. Zu den Unternehmen, die Gebrauch von Bitbucket machen, zählen neben Atlassian:
DHL
PayPal
Tesla Motors
The New York Times
Auch die Open-Source-Projekte Eigen und OGRE sind bei Bitbucket gehostet.
