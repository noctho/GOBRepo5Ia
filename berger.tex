\section{GitHub}

\subsection{Einführung}
Github ist im Grunde eine Website, auffindbar unter http://github.com. GitHub erlaubt es, online Dateien anhand Versionsverwaltungssystems Git
abzuspeichern. Daher auch der Name.

Meist wird GitHub für Softwareprojekte benutzt und normalerweise auch in Gruppen zu mehreren Personen. Das ist jedoch nicht bedingt. Viele Benutzer nutzen GitHub auch z.B. als Backup eigener Dateien oder einfach um Dateien (welche auch gar nichts mit Software zu tun haben) online anderen Nutzern zur Verfügung zu stellen.

\subsection{Geschichte GitHubs}
Erschienen ist GitHub erstmals im Februar 2008. Github, Inc., welche die Website nun betreibt, hieß damals noch Logically Awesome. Entwickelt wurde es von Chris Wanstrath, PJ Hyett und Tom Preston-Werner, wobei sie Ruby on Rails und Erlang benutzten. Eigentümer der Website ist die GitHub, Inc., welche seit 2007 besteht. Ihren Sitz hat sie in San Francisco.

Am 24. Februar 2009 wurde bekanntgegeben, dass innerhalb eines Jahres eine Zahl von 46.000 öffentlichen Repositories erreicht wurde. Davon wurden allein 17.000 im Januar desselben Jahres erstellt. Geforkt wurden zu dieser Zdeit 6.200 Repositories und gemerget wurden 4.600. Einige Monate darauf, am 5. Juli 2009 wurde dann die 100.000-User-Marke überschritten. Im selben Monat wurde weiters bekanntgegegeben, dass der Dienst nun 135.000 Repositories hostet, wovon 90.000 öffentlich sind.

Die Seite wuchs weiterhin ziemlich schnell: Am 25. Juli 2010 wurden bereits eine Million Repositories auf GitHub gehostet. Nicht ganz ein Jahr später, am 20. April 2011 waren es dann bereits doppelt so viel. 2013 wurde die 10-Millionen-Marke überschritten.

Im Folgejahr wurde die Seite in Russland aufgrund von Suizid-Referenzen und in Indien aufgrund von Repositories der IS blockiert. Indien machte die Website jedoch im Jahr darauf wieder der Öffentlichkeit zugänglich.

Am 26. März 2015 griff die chinesische Regierung die Website mit einer groß ausgelegten DDoS-Attacke (Distributed Denial of Service) an. Grund war ein Projekt, welches die New York Times für chinesische Leser bereitstellte. Es hostete von der chinesischen Regierung gesperrte Webseiten, sodass diese in China trotzdem lesbar wurden. Die chinesische Suchmaschine Baidu wurde so manipuliert und mit Schadcode versehen, dass Besucher der Website Teil der DDoS-Attacke wurden.

\subsection{GitHub}
GitHub ist, ähnlich Bitbucket, SourceForge und Google Code, ein Online-Hosting-Service. Anhand des Versionskontrollservices Git kann man Dateien auf GitHub laden, aktualisieren und holen. GitHub wird zum größten Teil für Open-Source-Software-Projekte benutzt.

Im Gegenteil zu beispielsweise SourceForge ist nicht das Projekt die zentrale Sammlung von Code, sondern ein vom User erstelltes Repository. Jeder User kann beliebig viele Repositories sammeln. Sofern diese öffentlich gemacht werden, sind sie kostenfrei. Private Repositories sind kostenpflichtig.

Ein Repository ist, wie bereits gesagt, eine Sammlung von Code. Ein Repository hat immer einen Besitzer, welcher mehrere Mitarbeiter festlegen kann. User, welche nicht Mitglied eines Repository sind, können von Git nicht direkt in dieses pushen. GitHub erlaubt es, Repositories zu "forken". Ein Fork ist eine Kopie eines Repository, mit welcher gearbeitet werden kann, ohne dass das ursprüngliche Repository verändert wird. Änderungen können mit Git von und nach einem Repository geladen werden. Möchte man dem ursprünglichen Ersteller eines geforkten Repository bitten, Änderungen in seinem originalen Repository aufzunehmen, so erstellt man einen Pull Request. Der Ersteller kann diesen dann annehmen oder ablehnen.

Neben einfachen Quellcode Hosting bietet GitHub auch weitere projektspezifische Dienste an:

\begin{itemize}
\item README-Dateien
Jedes Projekt kann eine README-Datei beinhalten, welche sich im Root der Projektstruktur befinden muss. Diese mit Markdown formatierte Datei kann GitHub dann auf der Projektseite anzeigen. Die Datei wird häufig zur Vorstellung des Projekts und für Installation/Benutzung sowie Regelung zur Mitarbeit benutzt.

\item Issue Tracking
Weiters stellt GitHub einen eigenen Issue Tracker für jedes Repository bereits. Dieser hat fundamentale Funktionalität: Issues können erstellt, diskutiert und geschlossen werden. Issues können Tags gegeben werden, um sie beispielsweise als Frage, Diskussion, Bug oder Feature markieren zu können (die Tags können beliebige Namen haben). Genauso können Issues Milestones zugewiesen werden. Auch können GitHub-User von anderen beauftragt werden, ein Issue zu lösen.

\item Projekt Wiki
Jedes Repository kann ein eigenes Wiki besitzen. Dieses kann vom Besitzer und den Mitarbeitern eines Repository bearbeitet werden. Es wird meist dazu genutzt, allgemeneie Funktionalitäten des Programms oder der Bibliothek zu beschreiben, sodass der Einstieg für einen Neuling einfacher ist.

\item Projekt Website
GitHub erlaubt das Hosten einer kleinen Website. Diese ist unter http://projektname.github.io erreichbar. Sie ist für eine kurze Vorstellung des Projekts gedacht.

\end{itemize}

\subsection{GitHub Enterprise}
GitHub Enterprise ist für Großbetriebe gedacht, welche GitHub firmenintern hosten möchten und nur ihre eigenen Projekte darauf nutzen wollen. GitHub Enterprise ist kostenpflichtig, der Einstiegspreis liegt bei 5.000\$ für 20 Personen.

\subsection{Gist}
Gists werden bereitgestellt, um kleine Code-Schnipsel schnell und einfach anzuzeigen. Anders als bei ähnlichen Diensten wie Pastebin können nicht nur einzelne Dateien geteilt werden, sondern auch mehrere Dateien bis hin zu ganzen Applikationen. Auch verwendet GitHub im Hintergrund nach wie vor Git, so dass auch die Gists unter einem Versionskontrollsystem laufen.
