\section{MediaWiki}
MediaWiki ist ein freies Softwarepaket zum Betrieb eines Wikis, das ursprünglich für die Wikipedia geschrieben wurde. Es ist das Wikisystem aller von der Wikimedia-Stiftung (Wikimedia Foundation) betriebenen Wikis sowie vieler anderer Wikis.
\subsection{Erklährung für wiki:}
Ein Wiki (hawaiisch für „schnell“), seltener auch WikiWiki oder WikiWeb genannt, ist ein Hypertextsystem für Webseiten, deren Inhalte von den Benutzern nicht nur gelesen, sondern auch online direkt im Webbrowser geändert werden können (Web-2.0-Anwendung). Das Ziel ist häufig, Erfahrung und Wissen gemeinschaftlich zu sammeln (kollektive Intelligenz) und in für die Zielgruppe verständlicher Form zu dokumentieren. Die Autoren erarbeiten hierzu gemeinschaftlich Texte, die ggf. durch Fotos oder andere Medien ergänzt werden (Kollaboratives Schreiben, E-Collaboration). Ermöglicht wird dies durch ein vereinfachtes Content-Management-System, die sogenannte Wiki-Software oder Wiki-Engine. Die bekannteste Anwendung von Wikis ist die Online-Enzyklopädie Wikipedia, welche die Wiki-Software MediaWiki einsetzt.

\subsection{Mediawiki allgemein}
MediaWiki ist eine freie Server-basierte Software, die unter der GNU General Public License (GPL) lizenziert ist. Es wurde entworfen, um auf einer großen Server-Farm eine Website zu betreiben, die Millionen Seitenzugriffe pro Tag erhält.
MediaWiki ist eine äußerst leistungsfähige, skalierbare Software und eine funktionsreiche Wiki-Implementierung, die PHP verwendet, um Daten zu verarbeiten und anzuzeigen, die in einer Datenbank wie MySQL gespeichert sind.
Auf den einzelnen Webseiten wird MediaWikis Wikitext-Format verwendet, so dass Anwender ohne Kenntnisse von XHTML oder CSS sie einfach bearbeiten und gestalten können.
Wenn ein Benutzer eine Bearbeitung auf einer Seite anlegt, schreibt MediaWiki es in die Datenbank, aber ohne die vorherigen Versionen der Seite zu löschen, so dass einfache Zurücksetzungen im Falle von Vandalismus oder Spam möglich sind. MediaWiki kann auch Bild-und Multimedia-Dateien verwalten, die im Dateisystem gespeichert werden. Für große Wikis mit vielen Benutzern, unterstützt MediaWiki Caching und kann leicht mit Squid-Proxy-Server-Software gekoppelt werden.

Seiten können leicht geändert werden, es können vorübergehend belanglose Sätze veröffentlicht werden, und eine Seite kann vorübergehend vollständig in einem Wiki zerstören werden. Dazu benötigt man keine Programmierkenntnisse.

Mediawiki ist eine freie Software, deswegen wird seitens der Entwickler, keinerlei Garantie gegeben oder irgend eine Gewährleistung jedweder Art gestellt.


\subsection{Worin MediaWiki nicht so gut ist.}
Weil es für offene Inhalte entwickelt ist, ist es oft nicht passend für Anwendungen, bei denen man den Zugang zu Teilen des Wikis beschränken will. (Though powering an entirely closed site, l	ike an internal company wiki, is not a problem.) 
MediaWiki ist entworfen, um Seiten mit hohem Zugriff wie Wikipedia zu managen. Es wurde für diesen Zweck optimiert und kann für kleinere Server weniger optimal sein, bei denen Plattenplatz oder RAM größere Einschränkungen sind als Bandbreite. 
MediaWiki ist keine typische Forum- (BBS-) oder Blog-Software, falls Sie genau das suchen. 

\subsection{Syntax}
kursiver Text : ''kursiv''
fetter Text : '''fett'''
