\section{Bonita BPM}
\subsection{Was ist Bonita BPM?}
Bonita PBM ist eine Open Source Software Suite f\"ur Prozessmodellierung,  Gesch\"aftsprozessmanagement (BPM) und Arbeitsablaufverwaltung (Workflow-Engine).  Damit ist es m\"oglich Prozesse zu automatisieren, dern Ergebnisse zu messen und diese Prozesse anschlie{\ss}ens zu verbessern. Die Suite besteht aus 3 Teilen: 
\begin{itemize}
 \item{\textit{Bonita Studio}: ein Studio in dem der Benutzer die Gesch\"aftsprozesse grafisch bearbeiten kann. Im Studio kann auch die vom Web aus erreichbare Gesch\"aftsanwendung generiert werden. Eine weitere Funktion des Bonita Studios ist es die Gesch\"aftsprozesse mit anderen Standards und Technologien zu designen.}
 \item{\textit{Bonita Portal}: ein Protal das dem Endbenutzer und den Administratoren mittels seiner Webseite eine \"ubersicht \"uber die Aufgaben und Prozessse zur Verf\"ugung stellt.}
 \item{\textit{Bonita Execution Engine}: eine Java Engine um programmiertechnisch mit den Prozessen zu interagieren}
\end{itemize}

\subsection{Wer braucht BonitaBPM?}
\begin{itemize}
\item{\textit{Angestellte:} Mit Bonita BPM haben Angestellte einen \"uberblick dar\"uber, welche Arbeiten anstehen und in welcher Reihenfolge diese abzuarbeiten sind.}
\item{\textit{Teamchefs:} Mit der Hilfe von Bonita BPM k\"onnen Teamchefs Aufgaben auf ihr Team verteilen.}
\item{\textit{Manager:} Durch die Aktivit\"atsanalyse die mit Hilfe von Diagrammen erstellt werden kann, k\"onnen Manager die Produktivit\"at der Firma erh\"ohen.}
\end{itemize}
\subsection{Funktionalit\"aten}
Bonita BPM bringt verschiedene Funktionen um das Gesch\"aftsprozessmanagement (BPM) von Projekten zu planen, zu entwickeln, ausf\"uhren und zu \"uberwachen:
\begin{itemize}
 \item{\textit{Live Updates der Prozesse}: Neue Versionen von Prozessen k\"onnen in der Produnktionsumgebung einfach aktualisiert werden}
 \item{\textit{Versionskontrolle der Prozesse}: W\"ahrend der modellierung der Prozesse k\"onnen verscheidene provisorische Versionen der Prozesse gespeichert und verwaltet werden}
 \item{\textit{Simulation der Prozesse}: Bonita BPM bietet die M\"oglichkeit dir Prozesse mit verschiedenen Parametern wie Kosten, Dauer, Ressourcenkonsum usw. zu simulieren.}
\item{\textit{Zuweisen und filtern von Aktivit\"aten}: Mit den eingebauten Instrumenten und Filtern ist es m\"oglich einer oder mehrerer Personen dynamisch und effizient Aktivit\"aten zuzuweisen.}
\item{\textit{Zentrales Repository}: Im zentralen Firmenrepository ist es m\"oglich alle Prozesse zu speichern, organisieren und archivieren}
\end{itemize}
\subsection{Weitere Eigenschaften}
Bonita BPM macht es m\"oglich das Gesch\"aftsprozessmanagement mit weiteren Bereichen eines Unternehmens zu verbinden. Zum Beispiel mit dem Kundenbeziehungsmanagement oder dem Ressourcenplanungsmanagement. Des weiteren bietet die Software Suite einen Kommentarfeed zu jedem spezifischen Fall um die Kommunikation innerhalb des Unternehmnens zu f\"ordern. Bonita BPM steht in den Sprachen Englisch, Franz\"osich und Spanisch zur Verf\"ugung. Die Interfaces k\"onnen durch ein eigenes Werkzeug aber in jede beliebige Sprache \"ubersetzt werden. Es ist weiters m\"oglich Prozesse in Echtzeit zu \"andern oder diese zu analysieren.