\documentclass[a4paper, 11pt]{article} 
\usepackage[ngerman]{babel} 
\usepackage[ansinew]{inputenc} 
\usepackage{graphicx} 
\usepackage{caption}
\usepackage{subcaption}
\usepackage{float}
\usepackage{url}
\usepackage{lipsum}
\usepackage{calc}
\usepackage{pgf-umlsd}

\begin{document}
\begin{center}
Bonita BPM
\end{center}
\newpage
\tableofcontents
\newpage
\section{Was ist Bonita BPM?}
Bonita PBM ist eine Open Source Software Suite f�r Prozessmodellierung,  Gesch�ftsprozessmanagement (BPM) und Arbeitsablaufverwaltung (Workflow-Engine).  Damit ist es m�glich Prozesse zu automatisieren, dern Ergebnisse zu messen und diese Prozesse anschlie�ens zu verbessern. Die Suite besteht aus 3 Teilen: 
\begin{itemize}
 \item{\textit{Bonita Studio}: ein Studio in dem der Benutzer die Gesch�ftsprozesse grafisch bearbeiten kann. Im Studio kann auch die vom Web aus erreichbare Gesch�ftsanwendung generiert werden. Eine weitere Funktion des Bonita Studios ist es die Gesch�ftsprozesse mit anderen Standards und Technologien zu designen.}
 \item{\textit{Bonita Portal}: ein Protal das dem Endbenutzer und den Administratoren mittels seiner Webseite eine �bersicht �ber die Aufgaben und Prozessse zur Verf�gung stellt.}
 \item{\textit{Bonita Execution Engine}: eine Java Engine um programmiertechnisch mit den Prozessen zu interagieren}
\end{itemize}

\section{Wer braucht BonitaBPM?}
\begin{itemize}
\item{\textit{Angestellte:} Mit Bonita BPM haben Angestellte einen �berblick dar�ber, welche Arbeiten anstehen und in welcher Reihenfolge diese abzuarbeiten sind.}
\item{\textit{Teamchefs:} Mit der Hilfe von Bonita BPM k�nnen Teamchefs Aufgaben auf ihr Team verteilen.}
\item{\textit{Manager:} Durch die Aktivit�tsanalyse die mit Hilfe von Diagrammen erstellt werden kann, k�nnen Manager die Produktivit�t der Firma erh�hen.}
\end{itemize}
\section{Funktionalit�ten}
Bonita BPM bringt verschiedene Funktionen um das Gesch�ftsprozessmanagement (BPM) von Projekten zu planen, zu entwickeln, ausf�hren und zu �berwachen:
\begin{itemize}
 \item{\textit{Live Updates der Prozesse}: Neue Versionen von Prozessen k�nnen in der Produnktionsumgebung einfach aktualisiert werden}
 \item{\textit{Versionskontrolle der Prozesse}: W�hrend der modellierung der Prozesse k�nnen verscheidene provisorische Versionen der Prozesse gespeichert und verwaltet werden}
 \item{\textit{Simulation der Prozesse}: Bonita BPM bietet die M�glichkeit dir Prozesse mit verschiedenen Parametern wie Kosten, Dauer, Ressourcenkonsum usw. zu simulieren.}
\item{\textit{Zuweisen und filtern von Aktivit�ten}: Mit den eingebauten Instrumenten und Filtern ist es m�glich einer oder mehrerer Personen dynamisch und effizient Aktivit�ten zuzuweisen.}
\item{\textit{Zentrales Repository}: Im zentralen Firmenrepository ist es m�glich alle Prozesse zu speichern, organisieren und archivieren}
\end{itemize}
\section{Weitere Eigenschaften}
Bonita BPM macht es m�glich das Gesch�ftsprozessmanagement mit weiteren Bereichen eines Unternehmens zu verbinden. Zum Beispiel mit dem Kundenbeziehungsmanagement oder dem Ressourcenplanungsmanagement. Des weiteren bietet die Software Suite einen Kommentarfeed zu jedem spezifischen Fall um die Kommunikation innerhalb des Unternehmnens zu f�rdern. Bonita BPM steht in den Sprachen Englisch, Franz�sich und Spanisch zur Verf�gung. Die Interfaces k�nnen durch ein eigenes Werkzeug aber in jede beliebige Sprache �bersetzt werden. Es ist weiters m�glich Prozesse in Echtzeit zu �ndern oder diese zu analysieren.
\end{document}