
\title{Microsoft Dynamics AX}
\author{Samuel Rufinatscha}

\maketitle
\begin{sloppypar}

\section{Microsoft Dynamics AX}


\subsection{Einführung}
Microsoft Dynamics® AX ist eine objektorientierte ERP-Lösung von Microsoft® für mittelständische Unternehmen und Großunternehmen, die es ermöglicht, Wandel und Veränderungen aus dem Geschäftsumfeld aufzunehmen und mitzugestalten und so den Geschäftserfolg voranzutreiben. Es handelt sich um eine leistungsstarke Lösung, mit der man in kürzester Zeit einen nachhaltigen Nutzen erzielen kann.
Dank der 36 verfügbaren Landesversionen eignet sich die Software insbesondere für den Einsatz in multinationalen Organisationen.
Microsoft Dynamics AX 2012 stellt den Beginn eines neuen Produktivitätszeitalters bei Unternehmenslösungen dar. Es handelt sich um die neueste Version der erfolgreichen ERP Software Reihe und beeindruckt mit einem deutlichen Zuwachs an Funktionalität, einer völlig neuen Agilität und einer Benutzeroberfläche, die überzeugt und Produktivität beim Anwender fördert.


\subsection{Ausgangssituation zur Entwicklung der ERP Software}
Zur Entwicklung dieser ERP Software wurden relevante aktuelle Business und IT-Anforderungen einbezogen: steigender globaler Wettbewerb, verbunden mit signifikantem Kosten- und Preisdruck, Organisationsänderungen, aber Prozessoptimierungen, das Entstehen von komplexen internationalen Netzwerken oder erhöhte Anforderungen durch steigende gesetzliche Kontrolle sind wichtige Trends, der sich Unternehmen stellen müssen.Es wird absolute Transparenz für schnelle und richtige Entscheidungen gefordert.
Viele Organisationen haben aber zunehmend Schwierigkeiten, mit diesen Trends bzw. Beweggründen Schritt zu halten, da ein altes und überholtes ERP System weder die nötige Flexibilität noch Prozessorientierung aufweist, um hier effizient agieren zu können.


\subsection{Die vier Prinzipien von AX}

\subsection{Leistungsstartk}

Folgende Eigenschaften tragen zur hohen Performanz der Microsoft ERP Software bei:
\begin{itemize}
\item Umfassende ERP Software-Funktionen sowohl für administrative als auch für operative Unternehmensbereiche
\item Reports können durch direkte Integration der Business Intelligence Funktionen des SQL-Servers rasch und effektiv verarbeitet werden.
\item Tasks eines Benutzers werden in einem zentralen Aufgabenbereich zusammengeführt und gegebenenfalls einer funktionalen Eingabewarteschlange zugewiesen.
\end{itemize}

\subsection{Agil}

Die Geschäftstätigkeit des Unternehmens kann durch eine Sammlung von einheitlichen Organisationsmodellen, überwacht, analysiert, bewertet und gegebenenfalls modifiziert werden.
Das Microsoft Dynamics ERP System löst zwei essentielle Fragen auf überzeugende Art und Weise: 
Wie gut bildet das ERP System die reale Welt ab und wie schnell lässt sie sich an das Unternehmen anpassen?

\subsection{Einfach}

Das Prinzip der Einfachheit spiegelt sich bei Installation, Implementierung, Anpassung, der Anwendungsoberfläche sowie bei Updates wieder, wobei die Grundlage zur Einfachheit die vertraute Benutzeroberfläche bildet.

\subsection{Global}

Eine globale Ausrichtung der aktuellen Microsoft ERP Software wird durch folgende Funktionen erreicht:

\begin{itemize}
\item Ab dem AX 2009-Release wurden Lokalisierungen für 38 Länder zur Verfügung gestellt.
\item Zentrale Stammdatenhaltung für Produkte und Geschäftspartneradressen wurden mit dem aktuellsten AX 2012 Release hinzugefügt.
\end{itemize}





\end{sloppypar}
