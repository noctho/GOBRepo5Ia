\section{Odoo}
\subsection{Was ist Odoo - mehr als OpenERP}
OpenERP heißt - wie bekannt - seit einigen Tagen Odoo. Was bedeutet das und warum hat es diese Namensänderung gegeben?
\subsection{Was bedeutet Odoo?}
Soweit wir wissen, hat das Wort Odoo keine wirkliche Bedeutung. Einige behaupten, dass Odoo das Kürzel von On Demand Open Object ist. Andere leiten es - mehr oder weniger direkt – von On Demand (SaaS) Offer from OpenERP ab. Die offizielle Verlautbarung zum Namen Odoo besagt, dass Firmen mit mehr \glqq O\grqq \space im Namen einen höheren Marktwert haben und deshalb sind 3 Os ein gutes Omen!

Ganz sicher aber gibt es einen sehr präzisen und praktischen Grund für die Einführung des Namens Odoo, nämlich den Wunsch, die Entwicklung vom einfachen und reinen ERP zu einem umfassenderen Angebot von Dienstleistungen und Tools für Unternehmen zu verdeutlichen. Dieses wird in der Modalität SaaS geliefert bzw. in der Version open source community edition zur Verfügung gestellt. Odoo ist, wie es Fabien Pinckaers, Gründer von OpenErp in einem Post geschrieben hat, eine Suite von Business Apps und kein einfaches ERP.
\subsection{Website Builder und e-Commerce: Echte Neuheiten von Odoo.}
Wer schon Erfahrung mit den vorhergehenden Versionen von OpenERP haben sollte, dem werden die beiden echten Neuheiten von Odoo auffallen, d.h. der Website Builder (ein integriertes, wenn auch noch ausbaufähiges CMS) und der mit dem ERP integrierte e-Commerce Motor (die Online-Verkäufe sind also direkt mit der Lieferantenverwaltung, den Kunden, dem Einkauf, dem Verkauf, dem Lager, dem Rechnungswesen usw.) verbunden. Alle anderen Funktionen (Projektmanagement, Personalverwaltung usw.) waren schon in OpenERP vorhanden, und zwar als Module (ab einem gewissen Zeitpunkt dann als App, für die Version SaaS), die bei Bedarf installiert werden konnten.
\subsection{Odoo wird weiterhin mit Open Source Lizenz ausgegeben.}
Odoo kann weiterhin frei genutzt werden. Sie müssen nur das Paket von der Webseite herunterladen, können Odoo aber auch in Modalität SaaS benutzen, wobei Sie direkt die Server von OpenErp SA verwenden (für nur zwei Abnehmer ist die Dienstleistung kostenfrei). In beiden Fällen benutzen Sie exakt die gleiche Software. Die neue Aufgabe von Odoo ist es, dem Unternehmen ein einziges Tool zu geben, das auf alle Firmenbedürfnisse 
antwortet (ERP aber auch Webseiten und e-Commerce), ohne verschiedene Software zu benutzen, die untereinander integriert werden müssen.

Wenn man die historische Entwicklung von OpenErp nach Odoo verfolgt, dann hat bei OpenERP die Erneuerung des herkömmlichen ERP schon ab der Version 7.0 begonnen. Dabei wurden Möglichkeit der Zusammenarbeit im Sinne der Social Media eingefügt, um die Leistungen dort zu steigern, wo das Teilen von Geschäftsinformationssystemen (E-Mail, File-sharing...) verhindert wurde.

Im Klartext: Odoo ist genau das, was die OpenERP Version 8 gewesen wäre, allerdings mit den neuen Zusatzmodulen. So hat man also den Übergang zu einer noch offeneren und auf integrierter Zusammenarbeit beruhenden Lösung.
\subsection{Hinweis für die Entwickler.}
Ein wichtiger technischer Hinweis für die Entwickler: Mit dem Übergang zur Version 8 erfolgt die Transition von Launchpad nach GitHub. Die Terminologie ändert sich: Es wird \glqq master\grqq \space anstelle von \glqq trunk\grqq \space verwendet. Der Übergang zu GitHub deshalb, weil Git schneller und GitHub besser organisiert ist. Außerdem bleibt Odoo, obwohl dem Namenswechsel das Wort Open zum Opfer gefallen ist, in jedem Fall Open, weil für den Download der verfügbaren Community Edition weiterhin die Lizenz GNU Affero General Public License bzw. GNU AGPL verwendet wird.
\subsection{Was ist die Aufgabe von Odoo?}
Wenn wir also das \glqq warum\grqq \space von Odoo kurz zusammenfassen wollen, dann können wir mit Sicherheit sagen, dass Odoo dem Unternehmen ein einziges und integriertes Tool gibt, das allen Firmenbedürfnissen entspricht (ERP, aber auch Webseitengestaltung und e-Commerce), ohne dass unterschiedliche Software verwendet wird, die anschließend integriert werden muss.
