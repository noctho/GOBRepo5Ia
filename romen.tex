\section{Activiti}
Activiti ist ein freies Workflow-Management-System mit dem man Businessprozesse definieren und ausführen kann. Es ist in Java geschrieben und damit Plattform unabhängig. Es ist ausgerichtet auf Geschäfts Leute, Entwickler und System Administratoren. Die Activiti Engine ist leight weight und auf Java Entwickler ausgelegt.

\subsection{Übersicht}
In folgender Grafik sind alle Teile der Activiti Software in den nächsten Kapitel werden die wichtigsten erklärt.
\begin{figure}[htbp]
	\centering
		\includegraphics[width=0.45\textwidth]{./components_overview.png}
	\caption{Teile der Activiti Software}
	\label{fig:ActivitiSWTeile}
\end{figure}

\subsection{Activiti Engine}
Die Activiti Engine ist das Herzstück des Projekts. Es ist eine Business Process Model and Notation 2 Engine mit API für Java. Es gibt dem Entwickler die Möglichkeit mit Listener bei bestimmten Ereignissen eigenen Java Code einzubauen um so mehr technische Details ins Diagramm zu bringen. Des weiteren können eigene Aktivitäten für das Diagramm definiert werden. Die ganze Engine ist von Anfang an darauf ausgelegt Cloud fähig zu sein.

\subsection{Activiti Explorer}
Der Activiti Explorer ist eine Webanwendung mit der das System Verwaltet werden kann. Sie ermöglicht alles vom erstellen neuer Aufgaben bis zum Abschließen jener. Es können Arbeiter verwaltet werden und deren Aufgaben ebenfalls. Für User zeigt die Webanwendung ihre aktuellen Aufgaben an und welche sie als nächstes machen müssen. Die Anwendung führt dabei History wer wann was macht.
\begin{figure}[htbp]
	\centering
		\includegraphics[width=0.45\textwidth]{./activiti-explorer-tasks.png}
	\caption{Der Activiti Explorer}
	\label{fig:ActivitiExplorer}
\end{figure}

\subsection{Activiti Modeler}
Der Activiti Modeler ist eine Webanwendung die es ermöglicht Grafische BPMN 2.0 Diagramme zu erstellen die ausgeführt werden können von der Activiti Engine.

\subsection{Activiti Designer}
Der Activiti Designer ist im Grunde das selbe wie der Activiti Modeler, es ist aber ein Eclipse plugin. Entwickler können damit direkt in Eclipse BPMN Diagramme für ihre Programme erstellen.
\begin{figure}[htbp]
	\centering
		\includegraphics[width=0.45\textwidth]{./activiti-designer.png}
	\caption{Der Activiti Designer}
	\label{fig:ActivitiDesigner}
\end{figure}
