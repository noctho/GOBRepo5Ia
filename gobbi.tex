\section{JIRA}
\begin{figure}
\subsection{Was ist JIRA?}
	JIRA ist eine von Atlassian entwickelte Projektmanagementsoftware die unter anderem auch Issuetracking und Statusverfolgung bereitstellt. JIRA wird nicht nur im Bereich der Softwareentwicklung verwendet da es auch Anforderungs und Aufgabenmanagement ermöglicht.
\subsection{Geschichte}
	JIRA wird seit 2002 von Atlassian Inc. entwickelt und betreut. Der Name kommt vom japanischen Wort \textbf{gojira} für Godzilla um bezug auf den bekannten Issuetracker Bugzilla zu nehmen. 
\subsection{Eigenschaften}
JIRA ist in Java programmiert und ist somit plattform unabhängig. Die GUI wird über ein Web-interface bereitgestellt und ist daher leicht erreichbar. JIRA ist nur für offizielle non-profit organisationen und open source Projekte frei verfügbar. Wenn eine Lizenz erworben wird erhält man eine Kopie des Quellcodes und darf diese für eigene Änderungen verwendet werden aber auf keinen Fall weitergegeben oder verkauft werden.
\subsection{}
\subsection{Nutzer}
	JIRA wirbt mit bis zu 25 000 aktiven Nutzern in über 122 Ländern. Bekante Firmen die JIRA verwenden sind Beispielsweise: NASA, BMW, Cisco, Adobe, \dots
